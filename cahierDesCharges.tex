\documentclass[a4paper]{article}

%% Language and font encodings
\usepackage[francais]{babel}
\usepackage[utf8x]{inputenc}
\usepackage[T1]{fontenc}

%% Sets page size and margins
\usepackage[a4paper,top=3cm,bottom=2cm,left=3cm,right=3cm,marginparwidth=1.75cm]{geometry}

%% Useful packages
\usepackage{amsmath}
\usepackage{graphicx}
\usepackage[colorinlistoftodos]{todonotes}
\usepackage[colorlinks=true, allcolors=blue]{hyperref}

\title{Cahier des charges}
\author{Benoit FAGET, Rémy MAUGEY, Oussama MEHDAOUI,\\Timothée JOURDE et Carlos NEZOUT}

\begin{document}

\maketitle
\newpage

\tableofcontents
\newpage

%% \begin{abstract}
%% Your abstract.
%% \end{abstract}

\section{Présentation du projet}
\subsection{Contexte}
Le musée des beaux-arts de Bordeaux ne peut pas exposer la majorité des œuvres de sa collection, son espace d’exposition étant insuffisant. Le musée souhaiterait trouver un moyen de valoriser ses œuvres restées en réserve, invisibles aux visiteurs.
\subsection{Objectif}
Nous voulons permettre au musée de faire sortir ses œuvres restées en réserve lors d’expositions temporaires. Ces expositions seront composées d’œuvres choisies exclusivement par les visiteurs et les internautes à l’aide d’une application web permettant de voter pour ses œuvres favorites parmis celles proposées par le musée. Cette application peut être proposée à la fin d’une visite en affichant la page web sur une des tablettes du musée, sur le site web du musée et être largement diffusée et partagée sur les réseaux sociaux par le musée. Tout ce qui a trait au vote (mise en place, œuvres proposées, durée, suivi et gestion) doit pouvoir être géré intégralement en interne par le personnel du musée.
\subsection{Description de l'existant}
Le musée dispose d’un espace libre suffisant pour recevoir une exposition temporaire. Il dispose également d’un site web sur lequel les photographies des œuvres en réserves sont disponibles, d'une base de données Mobydoc exportable et est présent sur les réseaux sociaux.
\subsection{Critères d'acceptabilité du produit}
L’application doit permettre au musée de :
\begin{itemize}
\item commencer/terminer un vote;
\item désigner les œuvres à soumettre au vote;
\item déterminer la durée et la description d'un vote;
\item suivre et gérer un vote en cours;
\item afficher les résultats d'un vote en cours ou terminé.
\end{itemize}
L’application doit permettre à l'utilisateur de voter sans authentification pour ses œuvres favorites.
%% Commentaires : revoir la définition fonctionnelle donnée / remplacer "internaute" par un autre terme.
\newpage

\section{Expression des besoins}
\subsection{Besoins fonctionnels}
D'un choix purement technique nous avons décidé de mettre en place deux plateformes de développement qui communiqueront avec la base de données. Côté musée, l'application n'est accessible que par le musée et permettra la gestion des votes par celui-ci. Côté utilisateur, l’application permet à l'utilisateur de soumettre son avis au musée en votant pour ses œuvres favorites parmis celles proposées. L’application est divisée en deux parties, musée (administration) et utilisateur (vote).
\subsubsection{Partie musée}
Il s’agit de la partie administrative de l’application, hébergée au musée et accessible uniquement en local.\\
Elle est découpée en plusieurs parties :
\begin{itemize}
\item une page d'accueil qui affiche la liste des sessions de votes terminées ou en cours et qui permet d'accèder aux pages de résultats et de création/modification;
\item une page de résultats par session permettant de consulter le nombre de votes pour chaque œuvre;
\item une page de création/modification permettant de créer/modifier une session.
\end{itemize}

\underline{Page d'accueil :}\\
%% Être plus intuitif sur l'état de la session (en cours ou non).
\begin{itemize}
\item Si un vote est en cours, la page d’administration informe de son avancée (temps restant, nombre de votes, classement des œuvres). Elle propose aussi de supprimer cette session à l’aide d’un bouton qui ouvre une fenêtre pop-up de confirmation lorsque l'utilisateur clique dessus. La page indique également la possibilité de modifier cette session en redirigeant vers la page de création.\\

\item Si aucun de vote n’est en cours, la page d’administration en recommande la création et propose de rediriger vers la page de création.\\

\item La page de création doit être accessible à l’aide d’un lien textuel.\\
\end{itemize}

\underline{Page de création/modification :}\\

\begin{itemize}
\item Si un vote est en cours, la page de création/modification propose de modifier la date de fin et l'ensemble des oeuvres (en vérifiant qu'elle n'est pas antérieure à la date actuelle) à l’aide d'un traitement ajax qui affichera le processus de modification et d'un pop-up de confirmation.\\ %TODO Benoit

\item Si un vote est en attente, la page de création/modification propose de modifier les dates de début, de fin (en vérifiant leur bonne chronologie) et la liste des œuvres à l’aide de multiples formulaires (pré-rempli à partir du vote actuel) et d'un pop-up de confirmation. Avant le popup de confirmation, la page doit vérifier que le format des champs est correct.\\

\item Si aucune session de vote n’a été créée, la page de modification est la même mais le formulaire est vide.\\

\item La page d’administration doit être accessible à l’aide d’un lien textuel.\\
\end{itemize}
\newpage

\subsubsection{Partie internaute}
Il s’agit de la partie visible de l’application web, libre d’accès pour les internautes. Elle est composée d’une unique page : une page d’accueil permettant à l’utilisateur de voter sans authentification pour ses œuvres favorites lorsqu'un vote est en cours ou de prendre connaissance des résultats lorsqu'un vote est terminé.\\

\underline{Page d'accueil :}\\
%% Remplacer les occurences de "vote" qui désignent le vote mis en place par le musée par un autre terme permettant de les différencier du vote de l'utilisateur pour une œuvre.

\begin{itemize}
\item Si un vote est en cours, la page d'accueil peut afficher un texte personnalisé comme une annonce ou une actualité. Elle affiche les vignettes des œuvres pour lesquelles l'utilisateur peut attribuer un vote. Si l'utilisateur clique sur la vignette d'une œuvre, une image agrandie accompagnée d'un texte explicatif comprenant le titre, le nom de l'artiste ainsi qu'une éventuelle description apparaît au premier plan, le reste de la page en arrière-plan est assombri. Un bouton permettant à l'utilisateur d'attribuer un vote est situé sous l'image et le texte. Si l'utilisateur clique sur le bouton, le vote est comptabilisé et l'on revient à la page en arrière plan.\\
%% Système d'étoiles, 1 ou 5, actualisation ou ajax (cookies) => avorté !
%% Affichage des œuvres en fonction du classement ?

\item L'internaute est limité à un seul vote par œuvre et par session (Il peut voter une deuxième fois pour une œuvre qui figure dans une session de vote différente). Cette limitaion est géré côté client par le localStorage.\\

\item Si un vote est terminé, la page d'accueil peut afficher un texte personnalisé comme une annonce ou une actualité. Elle affiche les vignettes des œuvres sélectionnées lors du précédent vote ainsi que les vignettes des deux œuvres ayant rassemblé le plus et le moins de votes.\\

\item S'il n'y a encore jamais eu de vote, la page d'accueil peut afficher un texte personnalisé comme uUn mode d’emploi pour la partie musée doit accompagner l’application à sa livraison.ne annonce ou une actualité.\\
\end{itemize}

\newpage

\subsection{Besoins non fonctionnels}
L'application web doit être portable: compatible avec les principaux navigateurs web (Firefox, Chrome, Edge) et l'affichage doit s'adapter à la largeur d'écran du client.\\

\section{Contraintes}
\subsection{Coûts}
Ce projet entrant dans le cadre d'un enseignement universitaire, aucun budget n'a été alloué. Il faut cependant prévoir un coût en ce qui concerne l'hébergement de l'application web.
\subsection{Délais}
Ce projet doit complété pour le 26 avril 2017 afin d'être soumis à évaluation par l'université. Cependant, aucun délai n'a été fixé pour un rendu de l'application web au musée.
\subsection{Autres contraintes}
Ce projet se concentre sur la partie « développement » de l’application web et n’aborde pas la partie « graphique ». Au même titre que le site web du musée, l’application doit être hébergée par la mairie de Bordeaux. Nous sommes aussi dépendants du musée en ce qui concerne la récupération de sa base de données présente sur Mobydoc.

\section{Déroulement du projet}
\subsection{Plannification}
Un cahier des charges corrigé doit être proposé pour le 09 février 2017. Une version finale du cahier des charges accompagnée de preuves de concept doit être proposée pour le 16 février 2017, cela implique une prise en main de Lumen, une version plus légère du framework Laravel. L'application web terminée doit être rendue pour le 26 avril 2017.

\newpage
\section{Implémentation}
Dans notre prototype les parties client et administrateur sont dans la même application. cela facilite le développement des deux parties qui utilisent du code commun. Ces parties sont facilement séparables pour le déploiement.
\subsection{Choix technologiques}
Les deux applications sont des applications web basées sur le framework Lumen. Elles utilisent une base de donnée de type MySQL locale ou distante suivant la configuration.

\subsection{Lumen}
Les applications sont divisées en plusieurs parties dans Lumen:
\begin{itemize}
\item La configuration du site (accès base de donnée) et des routes;
\item Les modèles Lumen d'accès à la base de donnée;
\item Les contrôleurs HTTP (réponses aux requètes HTTP);
\item Les vues (pages web servies à l'utilisateur par l'application).
\end{itemize}

\subsection{Base de donnée}
Le schéma de la base de donnée est le suivant:\\
\newline
\includegraphics[keepaspectratio=true, width=\textwidth]{test.png}

Ce schéma permet d'éviter la dupplication des oeuvres et des auteurs à travers les sessions. En effet, une même oeuvre dans la base de donnée peut être rattachée à plusieurs sessions.\\
Les applications accèdent à la base de donnée grâce à des modèles Lumen qui permettent de restranscrire en PHP le modèle précédent de base de donnée avec ses relations.
